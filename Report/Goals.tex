\section{Project Goals}
Sentiment Analysis is a natural language processing technique that aims to identify the sentiment present in a given text, in order to classify its polarity. Such technique finds application in several fields, including the news. While numerous works have been done on sentiment analysis of news articles \cite{News_1, News_2}, only a few put their attention of news comments. However, the set of comments under online articles constitutes an interesting dataset with possibilities of extracting useful information.
Only two studies are present about sentiment analysis of news comments, retrieved respectively from \textit{The Guardian} \cite{Guardian} and Chinese news websites \cite{Chinese}. Although both articles provide a detailed study of polarity of news comments, their only objective is to classify comments based on polarity by means of a properly trained model. No previous work has been found regarding the search for \textit{controversy} in such comments. 

\noindent The goal of this project is to determine whether an online article raised controversy. In order to do this, an index of controversy for each article was defined. Ultimately, the goal of this project is to uncover the most controversial categories of online news articles, highlighting the difference between the former and the most popular categories.
\section{Dataset}

The dataset used for the project is the \textit{New York Times Comments} dataset available on Kaggle \cite{Dataset}. As the dataset webpage states:

\begin{quote}
The data contains information about the comments made on the articles published in New York Times in Jan-May 2017 and Jan-April 2018. The month-wise data is given in two csv files - one each for the articles on which comments were made and for the comments themselves. The csv files for comments contain over 2 million comments in total with 34 features and those for articles contain 16 features about more than 9,000 articles.
\end{quote}

Among the available features, this project made use of the following:
\begin{itemize}
\item For the comment dataset: article ID, comment body, editors selection, recommendations, section name, desk and type of material.
\item For the article dataset: article ID, author, section, desk and type of material.
\end{itemize}

The dataset was easy to upload and manipulate thanks to the {\tt pandas} framework. However, more than half of the articles did not belong to any news section. This issue will be dealt more accurately in Sec. \ref{Resul}. 